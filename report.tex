%% LyX 2.2.3 created this file.  For more info, see http://www.lyx.org/.
%% Do not edit unless you really know what you are doing.
\documentclass[english]{article}
\usepackage[T1]{fontenc}
\usepackage[latin9]{inputenc}
\usepackage{float}
\usepackage{graphicx}

\makeatletter

%%%%%%%%%%%%%%%%%%%%%%%%%%%%%% LyX specific LaTeX commands.
%% Because html converters don't know tabularnewline
\providecommand{\tabularnewline}{\\}

\makeatother

\usepackage{babel}
\begin{document}

\title{The Sun is a Deadly Laser}

\title{Electrical Engineering Department Project, 2018 Spring Semester}

\author{Bill Chen, Alex Yao, Angela Wang, Callie Singer, Bennet BRown\\
\\
\\
\\
\\
\\
\\
\\
\\
\\
\\
\\
\\
\\
\\
\\
\\
\\
\\
\\
\\
\\
\\
\\
\\
\\
\\
\\
\\
\\
\\
}
\maketitle

\section*{Design}

\subsection*{System Overview: }

Columbia secondary school has requested a solar panel system that
will accommodate the needs of their school garden. Our solar panel
system will be located in front of Pupin Hall and consists of 12 solar
panels with dimensions 39.7 x 26.7 x 1.4 inches. Students from the
secondary school will come daily to collect their Windy Nation 12V
charged battery. They will also bring back their uncharged battery
from the previous day to be recharged. 

Our design has been greatly simplified to accomodate to the available
materials and simplifying the design process. However, an overview
of our calculations regarding the the materials can be found in the
calculations section. Our goal is to be able to power the Garden AC
System for 2 hours at 120V and 20 amps. 

\subsection*{The Solar Panel:}

We narrowed our panels down to two candidates:\\

\textbf{Renogy 100 Watt 12 Volt Polycrystalline Solar Panel}\\

\begin{tabular}{|c|c|}
\hline 
Maximum Power  & 100W\tabularnewline
\hline 
\hline 
Optimum Operating Voltage  & 17.8 V\tabularnewline
\hline 
Optimum Operating Current & 5.62 A\tabularnewline
\hline 
Weight: 16.5 lbs  & Dimensions: 39.7 x 26.7 x 1.4 inches\tabularnewline
\hline 
Cost  & \$105\tabularnewline
\hline 
\end{tabular}

\textbf{100 Watt Flexible Solar Panel with SunPower Solar Cells from
Windy Nation}\\

\begin{tabular}{|c|c|}
\hline 
Maximum Power & 100W\tabularnewline
\hline 
\hline 
Optimum Operating Voltage & 17.8V\tabularnewline
\hline 
Maximum Power Point & 5.62A\tabularnewline
\hline 
Weight 4.1 lbs (1.85 kg) & Module Dimension (L x W x H) (41.7\textquotedbl{} x 21.3\textquotedbl{}
x 0.1\textquotedbl{})\tabularnewline
\hline 
Cost & \$179.99\tabularnewline
\hline 
\end{tabular}\\

We can see from these two tables that the performance ratings of the
two are similar but the monocrystalline 100 Watt Solar Panel is flexible,
thinner and lighter. It also costs a lot more (179 vs 105). Due to
our needs for a long term power solution that will probably be mounted
permanently. We chose the polycrystalline solar panel to save on costs. 

\subsection*{Other Materials and Setup:}

As for other materials, we choose the following for simplicity and
compatibility after careful research. Batteries are chosen to 1) have
deep cycling compatibility, 2) be able to hold the necessary 4800
WHr energy needed to power the garden AC system.\\

\begin{table}[H]

\caption{Other parts}

\begin{tabular}{|c|c|c|}
\hline 
Batteries & 12V 100 Amp-Hour Deep Cycle AGM Sealed Lead Acid BatteryX6  & \$185x6=\$1110\tabularnewline
\hline 
\hline 
Inverter & VertaMax 3000 Watt 12V Pure Sine Wave Power Inverter DC to AC & \$416\tabularnewline
\hline 
Regulator & TrakMax 30L LCD MPPT 30A Solar Charge Controller Regulator  & \$200\tabularnewline
\hline 
\end{tabular}
\end{table}

The batteries will be connected in parallel. During charging, it will
be connected to the regulator which utilizes MPPT algorithm to maxmize
the power output of the solar cell array. The regulator will be connected
to the solar cells. The solar cells will be connected in parallel
as there will be very little distance between the regulator and the
cells: loss due to high amperage energy transfer is minimized. The
system is also more immune to the shading problem - one panel shaded
will not affect the whole string, a problem worth considering as our
system will be surrounded by buildings. 6 batteries will be able to
store a total of 7200 WHr of energy. Even though technically only
4 batteries are needed for our need of 4800 WHr, the extra two batteries
should be able to hold excess charge in times when the sun has more
available energy and can be used as backup power for rainly and the
winte days.\\

When discharging, the batteries will be hooked up to the power inverter
which boosts the 12V battery voltage to 120V, allowing the garden
AC system to draw the power.

\section*{Calculations:}

We calculated the amount of Solar Energy expected per square meter
in New York with Python. We first calculated the distance of Earth
from the Sun as a function. We then took into account of New York's
location and Earth's declination angle. Python source code of our
calculation is available on github {[}1{]} The permittivity of the
atmosphere to solar radiation was adjusted to be 0.75 to fit the lowerbound
of our model to available official data {[}2{]}. A quick reference
to other resources finds our assumption rational {[}3{]}. We assumed
that we are able to obtain around 6 hours of perfect sunlight everyday
as our conservative estimate. The advantage of using a model like
ours is that we can not only predict how many solar panels we would
need to charge our batteries to 4800 watt-hours, we can effectively
predict how much solar energy we can expect from the sun each and
everyday of the year. 

\begin{figure}[H]
\caption{Our solar energy graph for an entire year}

\includegraphics{\string"/Users/bill/Documents/untitled folder/Figure_1\string".png}
\end{figure}

Taking the lowest of the year: 2700 Watt-Hours/$m^{2}$, which roughly
occurs at the 350th day mark, and assuming the efficiency coefficient
of the solar panels to be 0.2 and taking into account each solar panel's
area: 0.687$m^{2}$, we can roughly approximate how many solar panels
we would need. 

Each panel output during the day of lowest sun exposure:
\[
0.687m^{2}\times0.2\times2700WHr/m^{2}=370.98WHr
\]
\[
4800WHr/370.98WHr=12.938Panels
\]

Since our estimate is extremely conservative (we did not take into
account of the left over charge we may have from the previous day
or any other time of the year and we have vastly underestimated daylight
time), we should be able to round down our estimation and 12 solar
panels should be sufficient to run our garden.

A few select points of our data is calculated here and put onto a
table:

\begin{table}

\caption{Select Data Points}

\begin{tabular}{|c|c|c|}
\hline 
Day & Total Energy in WHr & \# of hours of garden power\tabularnewline
\hline 
\hline 
1 & 4719 & 1.96\tabularnewline
\hline 
50 & 6875 & 2.86\tabularnewline
\hline 
100 & 10173 & 4.23\tabularnewline
\hline 
150 & 11716 & 4.88\tabularnewline
\hline 
200 & 11569 & 4.82\tabularnewline
\hline 
250 & 9755 & 4.06\tabularnewline
\hline 
300 & 6459 & 2.69\tabularnewline
\hline 
350 & 4436 & 1.84\tabularnewline
\hline 
\end{tabular}
\end{table}

\subsection*{Cost}

As for total cost Solar panels:
\[
105\times12=\$1260
\]

We add everything above together, obtaining:
\[
1260+416+200+1110=\$2986
\]

Along with installation fees and such, we are estimating a budget
proposal of around $\$3500$ for a relatively robust system.

\section*{References}

{[}1{]} https://github.com/billchen99/AOE\_SolarPanel/blob/master/solar\_alex1.py\\
{[}2{]} \emph{Solar Energy Resource Throughout New York}, http://www.asrc.cestm.albany.edu/perez/publications/\\
{[}3{]} \emph{Solar Energy To Earth http://energyeducation.ca/encyclopedia/Solar\_energy\_to\_the\_Earth}
\end{document}
